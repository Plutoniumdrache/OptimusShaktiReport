% !TEX spellcheck = en_US
% Collective Pitch Controller
%=================================================================================
Collective pitch control (CPC) adjusts the pitch for all 3 blades similarly.
The pitch control behavior has a high impact on the structural loads therefor on the life time of the wind turbine and thus on costs.
CPC can be implemented with a standard PI-Controller.
Main task of the CPC is to make the rotor area more permeable for the wind in order to reduce the power coefficient.
This is done by pitching the rotor blades in a less advantageous aerodynamic position.
With increasing wind speed the power output increases as well as the loads.
In order to keep the loads within an acceptable limit the power output of the wind turbine must be limited.

The pitch controller is only active in region 3, when the wind speed is above the rated wind speed as described in figure \ref{fig:control regions}.
In region 3 the pitch controller maintains rated speed and the generator torque controller rated torque. In the OPTIMUS Shakti wind turbine a gain scheduled PI controller is used to control the rotor speed.

The concept of gain scheduling is widely used and a common feature in blade pitch controllers.
With the use of gain scheduling the parameter of the controller are changed based on the operating point of the system. 



