% !TEX spellcheck = en_US
%=================================================================================
\chapter{Introduction and motivation (Soni)}

The report provides an overview of the dynamic controller for the OPTIMUS-Shakti-5MW wind turbine. Developed as part of the "Development of a Wind Turbine" module in the Wind Energy Engineering Master’s program at the University of Applied Sciences in Flensburg, this prototype aims to prepare students with the skills needed to develop a wind turbine. This year’s OPTIMUS turbine has a rated power of 5MW and a rotor diameter of 178 meters. Based on the tubular concept with a gearbox drivetrain, it is planned for onshore installation in Karnataka.
\\[16pt]
General objectives and requirements regarding the design of a dynamic controller will be discussed in the following chapter. It is important to highlight the challenges posed by operating in low wind speed areas. One critical aspect is the large rotor disc, which must perform efficiently under these conditions. The varying wind speeds result in dynamic loads that the control system must manage.
\\[16pt]
For the design of the advanced controller, we used a 3.4 MW reference wind turbine from the IEA Wind TCP Task 37 as a benchmark. This provided a solid foundation for developing a reliable control system that optimizes performance under various conditions.
\\[16pt]
The report covers essential topics in wind turbine design and optimization, including wind generation, a simple storage system with scenarios, tower bending stiffness, and peak shaving. It also addresses controller tuning. Challenges faced, generator speed control in Region 2.5, rated wind speed, and the mismatch between SLOW and FAST models are discussed. The report concludes with a summary of milestones achieved and an evaluation of the dynamic controller's integration into the OPTIMUS-Shakti-5MW wind turbine system.