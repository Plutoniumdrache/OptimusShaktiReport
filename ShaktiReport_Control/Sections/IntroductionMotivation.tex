% !TEX spellcheck = en_US
%=================================================================================
\chapter{Introduction and Motivation (Soni)}

The report provides an overview of the dynamic controller for the \gls{shakti} wind turbine. Developed as part of the "Development of a Wind Turbine" module in the Wind Energy Engineering Master’s program at the University of Applied Sciences in Flensburg, this prototype aims to prepare students with the skills needed to develop a wind turbine. This year’s Optimus turbine has a rated power of \SI{5}{MW}, a rotor diameter of \SI{178}{m} and a \SI{5}{kWh} battery storage. Based on the tubular concept with a gearbox drivetrain, it is planned for onshore installation in Karnataka, India.
\\
\\
General objectives and requirements regarding the design of a dynamic controller will be discussed in chapter \ref{chapter: controller design obj}. 
It is important to highlight the challenges posed by operating in low wind speed areas.
One critical aspect is the large rotor disc, which must perform efficiently under these conditions.
The varying wind speeds result in dynamic loads that the control system must manage.
Chapter \ref{chapter: controller design obj} contains as well the description of a tower damper design with the goal of optimizing the integration of a pitch actuator.
\\
\\
Support task that are not directly related to the wind turbine control system but that where important for the project progress are described in chapter \ref{chapter: support tasks}. 
These were the creation of an realistic wind field and a simple storage system dummy.
\\
\\
Chapter \ref{chapter:controller tuning} is dedicated to the controller tuning with the focus on steady states calculations, control parameter optimization that influence \gls{AEP} such as the minimum pitch angle.
\\
\\
The report discusses in chapter \ref{Challenges} different challenges that were faced during the project e.g. a mismatch between the used \gls{FAST} and \gls{SLOW} model. Furthermore it is explained how the team worked together and what major lessons were learned from the development project.
\\
\\
For the design of the advanced controller, a \SI{3.4}{MW} reference wind turbine from the IEA Wind TCP Task 37 \cite{IEA} was used as a starting point.
The reference model provided a solid foundation for developing a reliable control system that optimizes performance under various conditions.
%\\[16pt]
%The report covers essential topics in wind turbine design and optimization, including wind field generation, simple storage system scenarios simulations and control parameter tuning.
%Different challenges were faced during the development such as design of the rated wind speed and the mismatch of the \gls{SLOW} and \gls{FAST} model. 
%The report concludes with a summary of milestones achieved and an evaluation of the dynamic controller's integration into the OPTIMUS-Shakti-5MW wind turbine system.