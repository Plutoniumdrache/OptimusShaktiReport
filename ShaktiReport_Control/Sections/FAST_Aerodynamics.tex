% !TEX spellcheck = en_US
% mismatch of aerodynamics in FAST and SLOW
%=================================================================================
During development and the release of the first \gls{FAST} version of the \gls{WT} controller the following problem was encountered.
In order to validate the new controller version for \gls{FAST} a simulation was done and the results were compared to the expected behavior of the turbine with the \gls{SLOW} model of the developed \gls{WT}.
The findings were inconclusive.

The simulations showed for the same conditions, such as constant wind, no pitch activity and a matching number of DOFs a difference in power production of \SI{16}{\%}.
The \gls{FAST} model showed for rated wind conditions a \SI{16}{\%} lower power output compared to the \gls{SLOW} model.

After investigating several possibilities in \gls{SLOW} and \gls{FAST} the aerodynamics in \gls{FAST} were identified as the cause of the mismatch.
The rotor blades team provided a $Cp$ lookup table with the software QBlade which is used by the \gls{SLOW} model for deriving the correct aerodynamic power from the rotor.
Since \gls{FAST} was used for the load simulations the blade design in QBlade had to be exported. \gls{FAST} in this case is not using a $Cp$ look up table and is instead calculating the aerodynamics directly from the blade design.
The design is input as airfoil data and the corresponding position along the blade.
The load simulations in FAST were based on the AerodynV15 module.
Because QBlade is only supporting the export up to version AerodynV13 the conversion from AerodynV13 to V15 was done separately by the rotor blades team.
During this conversion an unidentified error occurred and lead to the faulty aerodynamics which lead to a difference in power production.

As a solution the change from the AerodynV15 module to the AerodynV14 module was approached.
This was done because the conversion from the QBlade output in AerodynV13 is compatible with the AerodynV14.
The change from one aerodyn version to another revealed that faulty airfoil data was exported by QBlade.
The files contained random NaN values as certain key values.
This problem was solved by correcting the wrong values after the export.
The change from AerodynV15 to AerodynV14 were able to solve the issue and make the \gls{SLOW} and \gls{FAST} simulation match.

The faulty exported data was not the cause for the mismatch of AerodynV15. This problem is still unclear. 

As in section \ref{RatedWindSpeed} described the rated wind speed was reduced during the design process from \SI{10.61}{m/s} to \SI{9.3}{m/s}.
Because of the higher wind speed used for the \gls{FAST} simulations up to that point the low aerodynamic power output was only identified quite late during the design process.
With the incorrect rated wind speed the \gls{WT} was apparently providing the rated power.
Which in truth the turbine was operating way above wind conditions to reach rated power.
Nevertheless the problem was identified in time and solved as a team effort of the loads, blades and control team. 
The described issue shows, that the validation of a model is from high importance and mistakes in totally different areas of responsibilities can lead to unexpected behavior later in the development phase. 
 