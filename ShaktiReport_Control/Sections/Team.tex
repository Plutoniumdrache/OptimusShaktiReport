% !TEX spellcheck = en_US
% Team
%=================================================================================
This section is describing how our team worked together during the duration of the \gls{shakti} Project. At the beginning it is worth to mentioning that we are also the same team in the lecture "Controller Design for Wind	Turbines and Wind Farms" \cite{SchlipfLecture} which helped a lot hence we were able to split the tasks of the lecture and the project together and to focus also on the lecture tasks because they are the main source and baseline on which we build up this project work.

Our weekly approach was to solve the lecture tasks first and as a team. The main benefit of this is that we all understanding the control theory of the task we are facing and being able to check each others work. The process in this project is not really different from the lecture work. We first meet after the weekly assembly meeting together and discuss what could be the next steps and what problems where phased during the last week. This internal meeting follows normally the weekly meeting with our supervisor Prof. Dr. David Schlipf. In this meeting new tasks and open questions were discussed. To solve the tasks the team sat together, often in front of a white board. After we understand the problem and a possible solution to it as far that the coding can begin we split the tasks and the coding is done mainly by a specific member of the team at home until someone is facing any issues. Than the problem solving loop regarding the coding starts again and is done by the whole team. 

As described above it is hard to assign specific roles to specific members of the team. Nevertheless we all had our niche areas in the project where each member spent more time and effort to: 
\begin{itemize}
	\item \textbf{Julius Preuschoff:} As the elected team lead of our group Julius has to attend the team leaders meetings on a weekly basis and is therefore mainly responsible for the communication with other teams and the management specially in a case when we figured out as design issue that could not be solved only by our group. During the controller design itself Julius had his focus mainly on the steady states calculation and the SLOW to FAST comparison.
	
	\item \textbf{Karan Soni:} Soni focused mainly on the Simulink implementation of the controller and the designing and testing of the parameters as well as the baseline TD.
	
	\item \textbf{Felix Lehmann:} Felix focused mainly on the Brute-force optimization approach for the parameters and the TD using the Lag-Compensator.   
\end{itemize} 
 
 To conclude and summarize the teamwork we worked really well as a team and the problems we phased where purely related to the technical work of the project and never personally. 
